
\section*{\LARGE Цель практической работы}
\addcontentsline{toc}{section}{Цель практической работы}

\textbf{Цель работы:} Отработка применения типизации событий и элемента
«Задача», а также маркеров действий при создании моделей процессов в
методологии BPMN.



\textbf{Постановка задачи:}\par
На основе выданного преподавателем варианта задания
\begin{itemize}
    \item Сформировать текстовое описание бизнес-процесса
    \item Построить бизнес-процесс в нотации BPMN,
    \item Подготовить презентацию для публичной защиты бизнеспроцесса, защитить полученную модель
\end{itemize}


\newpage

\section*{\LARGE Результат работы}

\addcontentsline{toc}{section}{Выполнение практической работы}
\section{Приготовление хачапури по аджарски}

На вход Приготовление хачапури по аджарски начинается с поступившего заказа помощнику повара, он начинает готовку теста.
Помощник смешивает ингредиенты, просеивает муку, ожидает поднятия тута, и разминает его пока оно не готово.
Далее идет подготовка теста поваром, он формирует тесто и приготавливает начинку, после обжаривает хачапури.
После шеф-повар выполнят сервировку

\img{img/img}{Приготовление хачапури по аджарски}

\img{img/img1}{Приготовление теста}
\newpage


\section*{ВЫВОД}
\addcontentsline{toc}{section}{ВЫВОД}
Построены и сохранены в
файле текстового формата текстовое описание бизнес-процесса, модели
бизнес-процесса, презентация бизнес-процесса, представленные
преподавателю в конце практического занятия в виде отчета.

\section*{СПИСОК ЛИТЕРАТУРЫ}
\addcontentsline{toc}{section}{СПИСОК ЛИТЕРАТУРЫ}
\begin{thebibliography}{}
    \bibitem{}  Материалы для практических/семинарских занятий: [url]
    \url{https://online-edu.mirea.ru/mod/resource/view.php?id=496092}
\end{thebibliography}
