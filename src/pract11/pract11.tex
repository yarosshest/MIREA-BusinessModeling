\graphicspath{{./img/}}

\section{Цель практической работы}
\textbf{Цель занятия:}
Отработка навыков по созданию моделей процессов в методологии BPMN.

\clearpage

\section{Выполнение практической работы}
\subsection{Модель процесса оеспечить оплату счета поставщика.}
Построим бизенс-процессы, указанные в методическом пособии, и исправим ошибки, допущенные при моделировании.


\begin{image}
	\includegrph{1}
	\caption{Измененный процесс «Обеспечить оплату счета поставщика»}
\end{image}

\clearpage
\newpage

\begin{image}
	\includegrph{2}
	\caption{Измененный процесс «Обработать заказ клиента»}
\end{image}


\begin{image}
	\includegrph{5}
	\caption{Измененный процесс «Процесс управления»}
\end{image}

\begin{image}
	\includegrph{3}
	\caption{Измененный процесс «Планирование работ по проекту»}
\end{image}

\clearpage
\newpage

\section*{ВЫВОД}
\addcontentsline{toc}{section}{ВЫВОД}
В результате практической работы №11 мы улучшили навыки создании моделей процессов в методологии BPMN.

\clearpage

\section*{СПИСОК ЛИТЕРАТУРЫ}
\addcontentsline{toc}{section}{СПИСОК ЛИТЕРАТУРЫ}
\begin{thebibliography}{}
	\bibitem{}  Материалы для практических/семинарских занятий: [url]
	\url{https://online-edu.mirea.ru/mod/resource/view.php?id=496092}
\end{thebibliography}
