
\section*{\LARGE Цель практической работы}
\addcontentsline{toc}{section}{Цель практической работы}

\textbf{Цель работы:} Построение функциональной диаграммы процесса.



\textbf{Постановка задачи:}\par
Построить концептуальную модель и сделать
декомпозицию концептуальной модели на основе текстового описания
процесса


\newpage

\section*{\LARGE Результат работы}

\addcontentsline{toc}{section}{Выполнение практической работы}
\section{Деятельность отдела продаж}

На вход Деятельности отдела продаж приходят:
\begin{itemize}
	\item Данные о клиенте
	\item Данные о продажах
	\item Товары на продажу
\end{itemize}

Механизмами являются:
\begin{itemize}
	\item Кассир
	\item Менеджер
	\item Склад
\end{itemize}

Управлением являются:
\begin{itemize}
	\item Законодательство
	\item Бизнес политика
\end{itemize}

На выходе будет:
\begin{itemize}
	\item Отгруженный товар
	\item Данные на получение товара
\end{itemize}

\img{img/img}{0 уровень Деятельность отдела продаж}

\newpage

На следующим уровне мы имеем 5 процессов:
\begin{itemize}
	\item Консультация клиента
	\item Проведение продажи
	\item Отгрузка товара
\end{itemize}

\img{img/img1}{1 уровень Деятельность отдела продаж}

\newpage

Подпроцесс Проведение продажи состоит из подпроцессов
\begin{itemize}
	\item Оформление заказа
	\item Контроль оплаты
	\item Продажа товара
\end{itemize}



\img{img/img2}{1 уровень Проведение продажи}

\newpage

\section*{ВЫВОД}
\addcontentsline{toc}{section}{ВЫВОД}
Построенные и сохраненные в
файле текстового формата структурно-функциональные диаграммы бизнеспроцессов,
представленные преподавателю в конце практического занятия.

\section*{СПИСОК ЛИТЕРАТУРЫ}
\addcontentsline{toc}{section}{СПИСОК ЛИТЕРАТУРЫ}
\begin{thebibliography}{}
	\bibitem{}  Материалы для практических/семинарских занятий: [url]
	\url{https://online-edu.mirea.ru/mod/resource/view.php?id=496092}
\end{thebibliography}
