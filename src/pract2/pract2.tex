
\section*{\LARGE Цель практической работы}
\addcontentsline{toc}{section}{Цель практической работы}

\textbf{Цель работы:} Построение функциональной диаграммы процесса.



\textbf{Постановка задачи:}\par
Построить концептуальную модель и сделать
декомпозицию концептуальной модели на основе текстового описания
процесса


\newpage

\section*{\LARGE Результат работы}

\addcontentsline{toc}{section}{Выполнение практической работы}
\section{Приготовление хачапури по аджарски}

На вход Приготовление хачапури по аджарски приходят:
\begin{itemize}
	\item Мука
	\item Яйца
	\item Соль
	\item Масло
	\item Сулугуни
	\item Вода
	\item Молоко
	\item Дрожжи
\end{itemize}

Механизмами являются:
\begin{itemize}
	\item Повар
	\item Полотенце
	\item Посуда
	\item Духовка
	\item Приборы
\end{itemize}

Управлением являются:
\begin{itemize}
	\item Рецепт
	\item Техника безопасности
\end{itemize}

На выходе будет xачапури по аджарски.

\img{img/img}{0 уровень Приготовление хачапури по аджарски}

\newpage

На следующим уровне мы имеем 5 процессов:
\begin{itemize}
	\item Приготовление теста
	\item Приготовление начинки
	\item Формирование теста
	\item Тепловая обработка
	\item Сервиловка
\end{itemize}

\img{img/img1}{1 уровень Приготовление хачапури по аджарски}

\newpage

Подпроцесс Приготовление теста состоит из подпроцессов
\begin{itemize}
	\item Смешать игридиенты
	\item Просеть муку
	\item Ожидание поднятия теста
	\item Размятие теста
\end{itemize}



\img{img/img2}{1 уровень Приготовление теста}

\newpage

\section*{ВЫВОД}
\addcontentsline{toc}{section}{ВЫВОД}
Построенные и сохраненные в
файле текстового формата структурно-функциональные диаграммы бизнеспроцессов,
представленные преподавателю в конце практического занятия.

\section*{СПИСОК ЛИТЕРАТУРЫ}
\addcontentsline{toc}{section}{СПИСОК ЛИТЕРАТУРЫ}
\begin{thebibliography}{}
	\bibitem{}  Материалы для практических/семинарских занятий: [url]
	\url{https://online-edu.mirea.ru/mod/resource/view.php?id=496092}
\end{thebibliography}
