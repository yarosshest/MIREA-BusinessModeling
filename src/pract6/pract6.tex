\section*{\LARGE Цель практической работы}
\addcontentsline{toc}{section}{Цель практической работы}

\textbf{Цель работы:} Формирование навыка проведения декомпозиции процесса в методологии IDEF0.


\textbf{Постановка задачи:}\par
На основе ранее выданного преподавателем варианта в практической работе 4:

\begin{enumerate}
	\item Поверить построенную функциональную диаграмму процесса на семантические ошибки.
	В случае обнаружения ошибок в использовании принципов построения моделей внести исправления в функциональную
	диаграмму и сформировать текстовый файл, в котором отразить все внесенные изменения;
	\item Выбрать любой подпроцесс в декомпозиции бизнес-процесса и построить следующий уровень детализации,
	руководствуясь тем, что входные и выходные потоки, а также механизм управления и исполнения уже заданы на более
	высоком уровне.
	Количество операций в детализируемом подпроцессе не может быть меньше 3 и ограничено 6;
	\item Сформировать табличное описание всех декомпозированных подпроцессов в файле текстового формата.
\end{enumerate}


\newpage

\section*{\LARGE Результат работы}

\addcontentsline{toc}{section}{Выполнение практической работы}
\section{Контекстная диаграмма}

Была построена контексная диограмма:

\img{img/img}{0 уровень}
\img{img/img2}{1 уровень}
\newpage
\img{img/img3}{3 уровень}

Дополнена данными блоками:
\img{img/img4}{2 уровень дополненый}

\section{Табличное описание}

Была построена таблица, всех декомпозированных подпроцессов.
\begin{table}[ht]
    \resizebox{\textwidth}{!}{%
        \begin{tabular}{|l|l|l|l|l|}
            \hline
            \begin{tabular}[c]{@{}l@{}}Название\\ подпроцесса\end{tabular}               & Краткое описание                                                                                                                                                                   & Исполнитель                         & Вход                                                                                                                                      & Выход                           \\ \hline
            \begin{tabular}[c]{@{}l@{}}Анализ данных\\  о приходе персонала\end{tabular} & \begin{tabular}[c]{@{}l@{}}Анализируются данные\\  о движении персонала\\  и на основе них создаться\\  информация о приходе персонала\end{tabular}                                & Специалист по управлению персоналом & Данные о движении персонала                                                                                                               & Данные о приходе персонала      \\ \hline
            \begin{tabular}[c]{@{}l@{}}Анализ увольнений\\  сотрудников\end{tabular}     & \begin{tabular}[c]{@{}l@{}}Анализируются данные о\\  движении персонала и на\\  основе них создаться\\  информация  о увольнений \\ сотрудников\end{tabular}                       & Специалист по управлению персоналом & Данные о движении персонала                                                                                                               & Данные о увольнений сотрудников \\ \hline
            Расчет показателей                                                           & \begin{tabular}[c]{@{}l@{}}Анализируются информация\\  о приходе персонала и \\  увольнений сотрудников\\  и на основе них создаться\\  показатели движения персонала\end{tabular} & Специалист по управлению персоналом & \begin{tabular}[c]{@{}l@{}}Средняя численность персонала;\\  Данные о увольнений сотрудников;\\  Данные о приходе персонала.\end{tabular} & Показатели движения персонала   \\ \hline
        \end{tabular}%
    }
\end{table}
\newpage
\begin{longtable}{|l|l|l|}
    \hline
    \textbf{Название диограммы/код} &
    \textbf{Наименование потока} &
    \textbf{Тип связи} \\ \hline
    \endfirsthead
%
    \endhead
%
    \begin{tabular}[c]{@{}l@{}}Учесть движение\\  персонала/A1\\ Рассчитать показатели\\  движения персонала/A2\end{tabular} &
    Данные о движении персонала &
    Выход-вход \\ \hline
    \begin{tabular}[c]{@{}l@{}}Учесть движение\\  персонала/A1\\ Рассчитать показатели\\  движения персонала/A2\end{tabular} &
    Приказ о переводе &
    Управление \\ \hline
    \begin{tabular}[c]{@{}l@{}}Учесть движение\\ персонала/A1\\ Рассчитать показатели\\ движения персонала/A2\end{tabular} &
    Приказ о приеме на работу &
    Управление \\ \hline
    \begin{tabular}[c]{@{}l@{}}Учесть движение\\ персонала/A1\\ Рассчитать показатели\\ движения персонала/A2\end{tabular} &
    Приказ об увольнении &
    Управление \\ \hline
    \begin{tabular}[c]{@{}l@{}}Рассчитать показатели\\ движения персонала/A2\\ Анализировать \\ показатели/A3\end{tabular} &
    Показатели движения персонала &
    Выход-вход \\ \hline
    \begin{tabular}[c]{@{}l@{}}Оформить прием\\  на работу/A11\\ о движении персонала\\  за период/A14\end{tabular} &
    Приказ о приеме на работу &
    Выход-вход \\ \hline
    \begin{tabular}[c]{@{}l@{}}Оформить перевод\\  кадров/A12\\ Сформировать данные\\  о движении персонала\\  за период/A14\end{tabular} &
    Приказ о переводе &
    Выход-вход \\ \hline
    \begin{tabular}[c]{@{}l@{}}Оформить увольнение\\  сотрудника/A13\\ Сформировать данные\\  о движении персонала\\  за период/A14\end{tabular} &
    Приказ об увольнении &
    Выход-вход \\ \hline
\end{longtable}
\newpage

\section*{ВЫВОД}
\addcontentsline{toc}{section}{ВЫВОД}
Построенные и сохраненные в
файле текстового формата структурно-функциональные диаграммы бизнеспроцессов,
представленные преподавателю в конце практического занятия.

\newpage
\section*{СПИСОК ЛИТЕРАТУРЫ}
\addcontentsline{toc}{section}{СПИСОК ЛИТЕРАТУРЫ}
\begin{thebibliography}{}
	\bibitem{}  Материалы для практических/семинарских занятий: [url]
	\url{https://online-edu.mirea.ru/mod/resource/view.php?id=496092}
\end{thebibliography}
