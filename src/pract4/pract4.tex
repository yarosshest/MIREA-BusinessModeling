
\section*{\LARGE Цель практической работы}
\addcontentsline{toc}{section}{Цель практической работы}

\textbf{Цель работы:} ознакомление с функциональными возможностями программного обеспечения по созданию бизнес-моделей
(процессов, осуществляемых различными сотрудниками И отделами организаций (предприятий, учреждений))
в методологии IDEF0.



\textbf{Постановка задачи:}\par
На основе выданного преподавателем варианта:

\begin{enumerate}
	\item Построить дерево узлов процесса, используя, например, SmartArt в текстовом редакторе.
	\item Построить контекстную диаграмму, детализацию контекстной диаграммы, детализацию одного из процессов, согласно выданному варианту, внеся данные об Авторе и проекте.
	\item Сформировать таблицу, где необходимо указать все Входы, Выходы, Механизмы и Управление.
	\item Выявить такие типы связей, как «Выход-Вход», «Обратная связь по входу», «Обратная связь по управлению», «Управление», «Выход-механизм», составить их список в таблице.
	\item Определить объект преобразования по типу: информационный или материальный, составить таблицу.
\end{enumerate}


\newpage

\section*{\LARGE Результат работы}

\addcontentsline{toc}{section}{Выполнение практической работы}
\section{Дерево узлов процесса}

Было построено дерево узлов процесса:

\img{img/tree}{0 уровень Дерево узлов процесса}

\newpage

\section{Таблица Входов, Выходов, Механизмов и Управления}
Была построена таблица, с указанием всех Входов, Выходов, Механизмов и Управления.

\begin{table}[ht]
    \resizebox{\textwidth}{!}{%
        \begin{tabular}{|l|l|l|l|l|}
            \hline
            \begin{tabular}[c]{@{}l@{}}Название\\ подпроцесса\end{tabular}               & Краткое описание                                                                                                                                                                   & Исполнитель                         & Вход                                                                                                                                      & Выход                           \\ \hline
            \begin{tabular}[c]{@{}l@{}}Анализ данных\\  о приходе персонала\end{tabular} & \begin{tabular}[c]{@{}l@{}}Анализируются данные\\  о движении персонала\\  и на основе них создаться\\  информация о приходе персонала\end{tabular}                                & Специалист по управлению персоналом & Данные о движении персонала                                                                                                               & Данные о приходе персонала      \\ \hline
            \begin{tabular}[c]{@{}l@{}}Анализ увольнений\\  сотрудников\end{tabular}     & \begin{tabular}[c]{@{}l@{}}Анализируются данные о\\  движении персонала и на\\  основе них создаться\\  информация  о увольнений \\ сотрудников\end{tabular}                       & Специалист по управлению персоналом & Данные о движении персонала                                                                                                               & Данные о увольнений сотрудников \\ \hline
            Расчет показателей                                                           & \begin{tabular}[c]{@{}l@{}}Анализируются информация\\  о приходе персонала и \\  увольнений сотрудников\\  и на основе них создаться\\  показатели движения персонала\end{tabular} & Специалист по управлению персоналом & \begin{tabular}[c]{@{}l@{}}Средняя численность персонала;\\  Данные о увольнений сотрудников;\\  Данные о приходе персонала.\end{tabular} & Показатели движения персонала   \\ \hline
        \end{tabular}%
    }
\end{table}


\newpage

\section{Таблица типов связий}

Была построена таблица типов связий.

\begin{longtable}{|l|l|l|}
    \hline
    \textbf{Название диограммы/код} &
    \textbf{Наименование потока} &
    \textbf{Тип связи} \\ \hline
    \endfirsthead
%
    \endhead
%
    \begin{tabular}[c]{@{}l@{}}Учесть движение\\  персонала/A1\\ Рассчитать показатели\\  движения персонала/A2\end{tabular} &
    Данные о движении персонала &
    Выход-вход \\ \hline
    \begin{tabular}[c]{@{}l@{}}Учесть движение\\  персонала/A1\\ Рассчитать показатели\\  движения персонала/A2\end{tabular} &
    Приказ о переводе &
    Управление \\ \hline
    \begin{tabular}[c]{@{}l@{}}Учесть движение\\ персонала/A1\\ Рассчитать показатели\\ движения персонала/A2\end{tabular} &
    Приказ о приеме на работу &
    Управление \\ \hline
    \begin{tabular}[c]{@{}l@{}}Учесть движение\\ персонала/A1\\ Рассчитать показатели\\ движения персонала/A2\end{tabular} &
    Приказ об увольнении &
    Управление \\ \hline
    \begin{tabular}[c]{@{}l@{}}Рассчитать показатели\\ движения персонала/A2\\ Анализировать \\ показатели/A3\end{tabular} &
    Показатели движения персонала &
    Выход-вход \\ \hline
    \begin{tabular}[c]{@{}l@{}}Оформить прием\\  на работу/A11\\ о движении персонала\\  за период/A14\end{tabular} &
    Приказ о приеме на работу &
    Выход-вход \\ \hline
    \begin{tabular}[c]{@{}l@{}}Оформить перевод\\  кадров/A12\\ Сформировать данные\\  о движении персонала\\  за период/A14\end{tabular} &
    Приказ о переводе &
    Выход-вход \\ \hline
    \begin{tabular}[c]{@{}l@{}}Оформить увольнение\\  сотрудника/A13\\ Сформировать данные\\  о движении персонала\\  за период/A14\end{tabular} &
    Приказ об увольнении &
    Выход-вход \\ \hline
\end{longtable}

\newpage

\section{Таблица типов связий}

Была построена таблица типов элементов.


\begin{table}[ht]
    \resizebox{\textwidth}{!}{%
        \begin{tabular}{|l|l|l|}
            \hline
            \textbf{Элемент нотации IDEF0} & \textbf{Наименование преобразуемого объекта} & \textbf{Тип}    \\ \hline
            Вход  & Резюме                           & информационный  \\ \hline
            Вход  & Анкетные данные                  & информационный  \\ \hline
            Вход  & Результаты тестирования          & информационный  \\ \hline
            Вход  & Результаты собеседования         & информационный  \\ \hline
            Вход  & Трудовая книжка                  & информационный  \\ \hline
            Вход  & Копии документов                 & информационный  \\ \hline
            Вход  & Заявление на замещение должности & информационный  \\ \hline
            Вход  & Заявление на увольнение          & информационный  \\ \hline
            Вход  & Заявление о приеме на работу     & информационный  \\ \hline
            Вход  & Данные тестирования              & информационный  \\ \hline
            Вход  & Средняя численность персонала    & информационный  \\ \hline
            Вход  & Данные опросного листа           & информационный  \\ \hline
            Внутренний поток               & Данные о движении персонала                  & инфрормационный \\ \hline
            Выход & Приказ о приеме на работу        & инфрормационный \\ \hline
            Выход & Приказ о переводе                & инфрормационный \\ \hline
            Выход & Приказ об увольнении             & инфрормационный \\ \hline
            Выход & Трудовой договор                 & инфрормационный \\ \hline
            Выход & Трудовая книжка                  & инфрормационный \\ \hline
            Выход & Показатели движения персонала    & инфрормационный \\ \hline
            Выход & Аналитическая отчетность         & инфрормационный \\ \hline
            Выход & Аналитическая записка            & инфрормационный \\ \hline
            Выход                          & Данные об освобождающейся вакансии           & инфрормационный \\ \hline
        \end{tabular}%
    }
\end{table}

\newpage

\section*{ВЫВОД}
\addcontentsline{toc}{section}{ВЫВОД}
Построенные и сохраненные в
файле текстового формата структурно-функциональные диаграммы бизнеспроцессов,
представленные преподавателю в конце практического занятия.

\section*{СПИСОК ЛИТЕРАТУРЫ}
\addcontentsline{toc}{section}{СПИСОК ЛИТЕРАТУРЫ}
\begin{thebibliography}{}
	\bibitem{}  Материалы для практических/семинарских занятий: [url]
	\url{https://online-edu.mirea.ru/mod/resource/view.php?id=496092}
\end{thebibliography}
