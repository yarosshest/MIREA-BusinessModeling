
\section*{\LARGE Цель практической работы}
\addcontentsline{toc}{section}{Цель практической работы}

\textbf{Цель работы:} Построение функциональной диаграммы процесса и ознакомление с
функциональными возможностями программного обеспечения



\textbf{Постановка задачи:}\par
Построить концептуальную модель и сделать
декомпозицию концептуальной модели на основе текстового описания
процесса


\newpage

\section*{\LARGE Результат работы}

\addcontentsline{toc}{section}{Выполнение практической работы}
\section{Формирование Технического проекта (ТП)}

На вход Формирование Технического проекта (ТП) приходят:
\begin{itemize}
	\item Утвержденный ЭП
	\item Уточненное ТЗ
	\item Уточняющие данные
\end{itemize}

Механизмами являются:
\begin{itemize}
	\item Уточняющие данные
	\item Программист
	\item Проектировщик
\end{itemize}

Управлением является стандарт.

На выходе будет технический проект.

\img{img/img}{0 уровень Формирование Технического проекта (ТП)}

На следующим уровне мы имеем 4 процесса:
\begin{itemize}
	\item Уточнение структуры и формы представления входных и выходных данных
	\item Разработка структуры программы
	\item Определение конфигурации технических средств
	\item Подготовка Пояснительной записки к ТП
\end{itemize}

Подпроцесс «Уточнение структуры и формы представления входных и
выходных данных» осуществляет бизнес-аналитик.
Входами подпроцесса  являются Утвержденное ТЗ, Утвержденный эскизный проект (ЭП),
Уточняющие данные.
Выходом подпроцесса (внутренним потоком) является Уточненная структура и форма представления данных.

Подпроцесс «Разработка структуры программы» осуществляет бизнесаналитик и программист, используя
Утвержденный ЭП и Уточненную структуру и форму представления данных.
Выходом подпроцесса (внутренним потоком) является Структура программа.

Подпроцесс «Определение конфигурации технических средств»
осуществляет программист и проектировщик исходя из Утвержденного ТЗ и
Структуры программы.
Выходом подпроцесса (внутренним потоком) является Конфигурация технических средств.

Затем Бизнес-аналитик на основе Уточненной структуры данных и их
формы, Структуры программы, Конфигурации технических средств реализует
подпроцесс «Подготовка Пояснительной записки к ТП».


\img{img/img1}{1 уровень Формирование Технического проекта (ТП)}

\section{Изготовление юбки}

На вход Изготовление юбки приходят:
\begin{itemize}
	\item Ткань
	\item Фурнитура
\end{itemize}

Механизмами являются:
\begin{itemize}
	\item Раскройное оборудование
	\item Швея-закройшица
	\item Швейная машинка
	\item Швейновышивальная машинка
\end{itemize}

Управлением является:
\begin{itemize}
	\item Выкройка
	\item Правила изготовления швейного изделия
\end{itemize}

На выходе будет платье.

\img{img/img2}{0 уровень Изготовления юбки}
\newpage

Раскрой материала осуществляется с использованием Раскройного оборудования и Выкройки.
В качестве преобразуемого материала выступает Ткань.
Выходом операции являются Детали изделия.

Сшивание деталей изделия осуществляется с использованием Швейной машинки.
Выходом операции является Собранное изделие.

Добавление фурнитуры осуществляется с использованием Швейновышивальной машинки.
Преобразуемыми ресурсами выступают Собранное изделие и Фурнитура (пуговицы, пряжка, аппликация).
Выходом операции является Платье

\img{img/img3}{1 уровень Изготовления юбки}