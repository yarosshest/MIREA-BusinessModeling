
\section*{\LARGE Цель практической работы}
\addcontentsline{toc}{section}{Цель практической работы}

\textbf{Цель работы:} Отработка применения типизации событий и элемента
«Задача», а также маркеров действий при создании моделей процессов в
методологии BPMN.



\textbf{Постановка задачи:}\par
На основе выданного преподавателем варианта задания
\begin{itemize}
    \item сформировать текстовое описание бизнес-процесса,определив роли (исполнителей), инициирующее и завершающее событие, тем самым определив границы бизнес-процесса
    \item Построить бизнес-процесс в нотации BPMN,
    \item Подготовить презентацию для публичной защиты бизнеспроцесса, защитить полученную модель
\end{itemize}


\newpage

\section*{\LARGE Результат работы}

\addcontentsline{toc}{section}{Выполнение практической работы}
\section{Организовать проведения противопожарной инспекции}
Начальник отдела формирует график проведения инспекций:
Проверяются сроки окончания заключений объектов из базы данных, на их основе формируется список объектов на проверку.
На основе списка и данных о сотрудниках, сопоставляются сотрудники и объекты.
Начальник отдела заверяет график проведения инспекций и передает его проверяющему.

Далее проверяющий проводит инспекцию, пиет заключение, выдает заключение.
Ответственный за ПБ получает заключение

\img{img/img}{Организовать проведения противопожарной инспекции}

\img{img/img1}{Сформировать график проведения инспекций}
\clearpage

\newpage

\section*{ВЫВОД}
\addcontentsline{toc}{section}{ВЫВОД}
Построены и сохранены в
файле текстового формата текстовое описание бизнес-процесса, модели
бизнес-процесса, презентация бизнес-процесса, представленные
преподавателю в конце практического занятия в виде отчета.
\newpage

\section*{СПИСОК ЛИТЕРАТУРЫ}
\addcontentsline{toc}{section}{СПИСОК ЛИТЕРАТУРЫ}
\begin{thebibliography}{}
    \bibitem{}  Материалы для практических/семинарских занятий: [url]
    \url{https://online-edu.mirea.ru/mod/resource/view.php?id=496092}
\end{thebibliography}
