
\section*{\LARGE Цель практической работы}
\addcontentsline{toc}{section}{Цель практической работы}

\textbf{Цель работы:} Отработка поиска ошибок в построенной модели бизнеспроцесса с последующим перепроектированием
модели процесса



\textbf{Постановка задачи:}\par
На основе выданного преподавателем варианта задания
\begin{itemize}
    \item Проанализировать представленную модель бизнес-процесса и смоделировать процесс заново.
\end{itemize}


\newpage

\section*{\LARGE Результат работы}

\addcontentsline{toc}{section}{Выполнение практической работы}
\section{Исправление ошибок}
Входе работы были переназванны, процессы и изменены соединения между ними.
\img{img/img}{Исправление ошибок}
\clearpage

\newpage

\section*{ВЫВОД}
\addcontentsline{toc}{section}{ВЫВОД}
Построены и сохранены в
файле текстового формата текстовое описание бизнес-процесса, модели
бизнес-процесса, презентация бизнес-процесса, представленные
преподавателю в конце практического занятия в виде отчета.
\newpage

\section*{СПИСОК ЛИТЕРАТУРЫ}
\addcontentsline{toc}{section}{СПИСОК ЛИТЕРАТУРЫ}
\begin{thebibliography}{}
    \bibitem{}  Материалы для практических/семинарских занятий: [url]
    \url{https://online-edu.mirea.ru/mod/resource/view.php?id=496092}
\end{thebibliography}
